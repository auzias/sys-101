\section{Memory Management}
  \begin{frame}
    \frametitle{Memory Management}
      Historically, no abstraction at all:
      \begin{itemize}
        \item RAM: OS (from 0 to 0xN) and user program (to 0xN to 0xMAX)\footnote{\label{fn:bug}These are dangerous: if a bug is present in the program, it can wipe the OS}
        \item RAM: user program (from 0 to 0xMAX), ROM: OS
        \item RAM: OS (from 0 to 0xN) and user program (to 0xN to 0xMAX), ROM: device drivers (see fn \ref{fn:bug})
      \end{itemize}
      Usually, only one sole program could run at a time. Unless...
  \end{frame}

  \begin{frame}
    \frametitle{No abstraction, several user programs}
    \begin{itemize}
      \item Swapping: the constraint is that only one program can be in the RAM at once. The OS can save into a file the whole memory and load up another back-up program state.
      \item Divide and conquer: IBM 360 split memory with a protection key to identify (and match) memory and programs.
      \begin{itemize}
        \item Static relocation: modify on the fly the program when it's loaded (i.e., JUMP 28 into JUMP 28+constant).
      \end{itemize}
    \end{itemize}
  \end{frame}

  \begin{frame}
    \frametitle{Memory abstraction: address spaces}
    \begin{block}{Problems to solve}
      \begin{itemize}
        \item Protection
        \item Relocation
      \end{itemize}
    \end{block}
    \begin{block}{Address space?}
      \begin{itemize}
        \item IP address
        \item Phone numbers
        \item Internet domain (.com, .net)
      \end{itemize}
    \end{block}
  \end{frame}

  \begin{frame}
    \frametitle{Memory abstraction: address spaces}
    Each process address space is mapped onto a different part of the physical memory.
    \begin{block}{Base and limit CPU registers}
      \begin{itemize}
        \item Base register: loaded with the physical address where the process begins.
        \item Limit register: loaded with the length of the process.
      \end{itemize}
    \end{block}
    Every time a memory reference is done, the CPU add the value contained in the base register, checks if it is equal or greater than the value in the limit register and cases a fault if needed.
  \end{frame}

  \begin{frame}
    \frametitle{Managing Free Memory}
    \begin{itemize}
      \item bitmaps: memory is divided as allocation units (from a few words to several kB). Each unit is map to a bit (1 if used, 0 if free).\footnote{smaller units allow a more precise allocation but the map consumes more memory}
      \item linked lists: memory is a whole block. When a process starts the process' address space is removed from the list till the process ends. The list is sorted by address. Several algorithm to allocate memory are: first fit, next fit, best fit\footnote{A bit wasteful as it tends to leave unusable tiny free spaces}, worst fit. This can be speed up with two lists: one for processes, one for holes -- and holes-list can then be sorted by size making best fit/first fit faster. Holes list can also be stored... in holes!
    \end{itemize}
  \end{frame}

  \begin{frame}
    \frametitle{Virtual Memory}
    Each program has its own address space, broken up into chunks called pages. These pages are mapped to physical memory, but not only.
    \begin{block}{Paging}
      Virtual addresses\footnote{program-generated addresses} form the virtual address space. When used, they go to an MMU\footnote{Memory Management Unit}, instead of the memory bus, that maps the virtual addresses onto the physical memory addresses. It aims to facilitate memory overlay.\\
      A page fault occurs when a process references an address that is not stored in the memory.
    \end{block}
  \end{frame}

  \begin{frame}
    \frametitle{Page Replacement Algorithms}
    \begin{itemize}
      \item optimal\footnote{easy to explain, impossible to implement on \emph{first} run}: remove the last referenced page. The OS has no way of knowing when each of the pages will be referenced next.
      \item not-recently-used: randomly removed one page of the lowest-numbered non empty class (classed thanks to R (rreferenced-flag) and M (modified-flag)):
      \begin{itemize}
        \item Class 0: not referenced, not modified
        \item Class 1: not referenced, modified
        \item Class 2: referenced, not modified
        \item Class 3: referenced, modified
      \end{itemize}
    \end{itemize}
  \end{frame}

  \begin{frame}
    \frametitle{Page Replacement Algorithms}
    \begin{itemize}
      \item FIFO: First-In, First-Out (can throw out a heavily used page).
      \item second-chance: to not throw out a heavily used page, R-flag is inspected on the oldest pages. The algorithm seek for old pages that has not been referenced in the most recent clock interval.
      \item clock: circular-list contains the pages, the oldest is always pointed. If its R-flag is 1, it is cleared and the next is pointed, when a page with R-flag cleared is found it is replaced.
    \end{itemize}
  \end{frame}

  \begin{frame}
    \frametitle{Page Replacement Algorithms}
    \begin{itemize}
      \item least-recently-used: recently heavily used page will probably be reused and pages not used for ages probably won't be. Longest unused page is replaced. Heavy algorithm: the list of pages must be updated at every memory reference. Special hardware does exist to do so.
      \item working-set: pages are loaded only on demand, not in advance. A process having a working-set won't cause page fault until it moves into another execution phase.
      \item WSClock: based on the clock algorithm and the working-set.
    \end{itemize}
  \end{frame}

  \begin{frame}
    \frametitle{Page Fault Handling}
    \begin{enumerate}
      \item The hardware traps to the kernel saving the program counter.
      \item ASM code is started to save volatile information\footnote{such as registers}.
      \item The OS discovers a page fault and seek for the needed page.
      \item Check if the address is valid and the protection. If not, process is killed. Is a page frame free? No: page replacement algorithm is run to select a victim.
    \end{enumerate}
  \end{frame}

  \begin{frame}
    \frametitle{Page Fault Handling}
    \begin{enumerate}\setcounter{enumi}{4}
      \item The found page is dirty, it is scheduled for transfer. Another process is run, in any event the page is marked as busy.
      \item The page is clean, the OS schedules a disk operation to bring the new page in.
      \item The disk interrupt indicates the new page arrival. Page tables are updated, the page is marked as being in normal state.
      \item Fault instruction is backed up the state it had when it began, the program counter is reset to point to that instruction.
      \item Faulting process is scheduled, the OS return to the routine that called it.
      \item The routine reloads the registers and other state information and returns to user space to continue execution.
    \end{enumerate}
  \end{frame}

  \begin{frame}
    \frametitle{Segmentation}
    \begin{block}{Segmentation}
      Division into different segments (or sections) of different types (code, data, array, stack).
      \begin{itemize}
        \item Each segments is contiguous.
        \item Memory is then address by two-part address: segment number and address within the segment.
        \item Segments simplify growing and shrinking data structures handling and sharing.
        \item Each segment can have a different protection.
      \end{itemize}

    \end{block}
  \end{frame}
